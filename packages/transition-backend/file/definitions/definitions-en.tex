\documentclass{article}
\usepackage[colorlinks = true,
            linkcolor = blue,
            urlcolor  = blue,
            citecolor = blue,
            anchorcolor = blue]{hyperref}

\usepackage{longtable}
\usepackage{tabulary}
\usepackage{multirow}
\usepackage{amsmath}
\usepackage[none]{hyphenat}
\usepackage[utf8]{inputenc}
\usepackage{array}
\usepackage{mathtools}
\usepackage{amsmath}
\usepackage{tabularx}
\usepackage{makecell}
\usepackage[legalpaper, landscape, top=0.5in, left=0.5in, right=0.7in, bottom=0.2in, includefoot]{geometry}

\begin{document}

\newcolumntype{C}[1]{>{\centering\arraybackslash}m{#1}}   %% centered
\newcolumntype{R}[1]{>{\raggedleft\arraybackslash}m{#1}}  %% right aligned
\newcolumntype{L}[1]{>{\raggedright\arraybackslash}m{#1}}  %% left aligned
\setlength\LTleft{0pt}
\setlength\LTright{0pt}
\setlength\LTcapwidth{\textwidth}
\setlength{\abovedisplayskip}{0.1pt}
\setlength{\belowdisplayskip}{0.1pt}
\renewcommand{\arraystretch}{1.2}
\newdimen\NetTableWidth

\title{Public transit • Definitions and symbols }

\author{Pierre-Léo Bourbonnais and students of the course CIV6708}

\section*{Public transit: Definitions and symbols}

\subsection*{Stops, stop nodes, stations and segments}

\noindent
  \NetTableWidth=\dimexpr
    \linewidth
    - 8\tabcolsep
    - 5\arrayrulewidth % if package array is loaded
  \relax

\begin{longtable}{%
    R{.15\NetTableWidth}%
    C{.08\NetTableWidth}%
    C{.04\NetTableWidth}%
    C{.08\NetTableWidth}%
    L{.65\NetTableWidth}%
}
\hline
\makecell[r]{Definition} & \makecell[c]{Symbol} & \makecell[c]{Unit} & \makecell[c]{Expression} & \makecell[l]{Description} \\
\hline
\hline
\endhead
\label{node}
\makecell[r]{Stop node} & \[q\] & - & - & Group of stop signs and/or transit platforms for boarding and alighting in the vicinity of each other and considered a potential transfer location when multiple lines serve the node. By instance, when multiple stop signs are located at each corner of an intersection, they are grouped in a single stop node. \\
\hline
\label{stop}
\makecell[r]{Stop} & \[s\] & - & - & Precise location of the platform centroid or stop sign allowing boarding and alighting of passengers for a unit in service on a line. Is part of a stop node. \\
\hline
\label{station}
\makecell[r]{Station} & \[S\] & - & - & Building making it easy to transfer between multiple stops. When more than one transit modes are served at a station, we call it an intermodal station. \\
\hline
\label{segment}
\makecell[r]{Segment} & \[l\] & - & - & Line path segment between two consecutive stops/stop nodes, represented by a specific route on the road or rail network. \\
\hline
\label{unique_stop_node}
\makecell[r]{Unique stop node} & \[q_u\] & - & - & Stop node served by a single line (one-way or bi-directional). \\
\hline
\label{multiple_stop_node}
\makecell[r]{Multiple stop node} & \[q_m\] & - & - & Stop node served by more than one line. \\
\hline
\label{unique_stop}
\makecell[r]{Unique stop} & \[s_u\] & - & - & Stop sign or platform served by a single line (one-way or bi-directional). \\
\hline
\label{multiple_stop}
\makecell[r]{Multiple stop} & \[s_m\] & - & - & Stop sign or platform served by more than one line. \\
\hline
\label{unique_segment}
\makecell[r]{Unique segment} & \[l_u\] & - & - & Segment served by a single line (one-way or bi-directional). \\
\hline
\label{multiple_segment}
\makecell[r]{Multiple segment} & \[l_m\] & - & - & Segment served by multiple lines. \\
\hline
\label{terminal}
\makecell[r]{Terminal} & \[s_T\] or \[q_T\]  & - & - & First or last stop/stop node for a line. Usually allows at least one transit unit to park during layover time. \\
\hline
\label{outbound_terminal}
\makecell[r]{Departure/Outbound terminal} & \[{s^{\prime}_T}\] or \[q^{\prime}_T\] & - & - & First stop/stop node on a line path. \\
\hline
\label{inbound_terminal}
\makecell[r]{Arrival/Inbound terminal} & \[{s^{\prime\prime}_T}\] or \[{q^{\prime\prime}_T}\] & - & - & Last stop/stop node on a line path. \\
\hline
\label{stop_time}
\makecell[r]{Stop-time} & \[s_t\] or \[q_t\] & - & - & Passage of a transit unit in service on a line at a particular stop/stop node, in a given direction and at a specific time (scheduled). Includes an arrival time when the unit arrives at the stop and a departure time when the unit leaves the stop.\\
\hline
\label{stop_sequence}
\makecell[r]{Stop-sequence} & \[s_{seq}\] or \[q_{seq}\] & - & - & A combination of a stop/stop node and a sequence number on a line path in one direction. A path includes a stop-sequence for each stop/stop node served. An outbound path with 4 stops (A, B, C and D) will have 4 stop-sequences: stop A at sequence 1, stop B at sequence 2, stop C at sequence 3 and stop D at sequence 4. The inbound path will have reversed stop sequences: stop D at sequence 1, stop C at sequence 2, stop B at sequence 3 and stop A at sequence 4.\\
\hline
\label{connection}
\makecell[r]{Connection} & \[c\] & - & - & Traveling of a transit unit on a particular path segment between two consecutive stop/stop nodes and according to a schedule. Includes a departure time from the previous stop and an arrival time at the next stop. \\
\hline
\end{longtable} 

\pagebreak
\subsection*{Vehicles, units and depots}

\begin{longtable}{%
    R{.25\NetTableWidth}%
    C{.08\NetTableWidth}%
    C{.04\NetTableWidth}%
    C{.1\NetTableWidth}%
    L{.5\NetTableWidth}%
}
\hline
\makecell[r]{Definition} & \makecell[c]{Symbol} & \makecell[c]{Unit} & \makecell[c]{Expression} & \makecell[l]{Description} \\
\hline
\hline
\endhead
\label{transit_unit}
\makecell[r]{Transit unit} & \[u\] & - & - & Set of cars, wagons, or trailers coupled together forming a train. A regular or articulated bus represents a single unit. \\
\hline
\label{vehicle}
\makecell[r]{Vehicle (trailer/car)} & \[y\] & - & - & Train car or trailer that is part of a unit. \\
\hline
\label{depot}
\makecell[r]{Depot} & \[G\] & - & - & Bus garage or rail depot where units and vehicles are stored and where maintenance is performed. Usually, the depot is also the departure and arrival point for vehicles and drivers assigned to service at the beginning and end of their runs. \\
\hline
\label{depot_unit_capacity}
\makecell[r]{Depot units capacity} & \[{N_u}_G\] & units & - & Number of units that can be simultaneously parked in the garage or depot. Care should be taken to specify whether this capacity includes spaces reserved for vehicles under maintenance and/or being charged. \\
\hline
\label{fleet_size}
\makecell[r]{Fleet size} & \[F\] & units & - & Total fleet (number of units) including units in service, reserve, and maintenance. \\
\hline
\label{reserve_fleet_size}
\makecell[r]{Reserve fleet size} & \[F_r\] & units & - & Number of units reserved to replace units that must be removed from circulation (incidents) and to add service during special events or severe congestion slowing planned service. \\
\hline
\label{maintenance_fleet_size}
\makecell[r]{Maintenance fleet size} & \[F_m\] & units & - & Number of units under repair or preventive maintenance. These units cannot be used for service. \\
\hline
\label{service_fleet_size}
\makecell[r]{Service fleet size} & \[F_s\] & units & \[F - F_m - F_r\] & Number of units that can be used to perform planned service during maximum peak period. \\
\hline
\label{fleet_use_ratio}
\makecell[r]{Fleet use ratio} & \[\mu_{F_u}\] & - & \[\frac{F_s + F_r}{F}\] & - \\
\hline
\label{reserve_fleet_ratio}
\makecell[r]{Reserve fleet ratio} & \[\mu_{F_r}\] & - & \[\frac{F_r}{F}\] & - \\
\hline
\label{maintenance_fleet_ratio}
\makecell[r]{Maintenance fleet ratio} & \[\mu_{F_m}\] & - & \[\frac{F_m}{F}\] & - \\
\hline
\label{service_fleet_ratio}
\makecell[r]{Service fleet ratio} & \[\mu_{F_s}\] & - & \[\frac{F_s}{F}\] & - \\
\hline
\label{vehicles_per_unit}
\makecell[r]{Number of vehicles per unit} & \[n_y\] & vehicles & - & Number of train cars or trailers coupled together forming the unit. \\
\hline
\label{required_units}
\makecell[r]{Required number of units} & \[n_u\] & units & - & Number of units required to perform service for a given period (usually on a particular line). \\
\hline
\label{maximum_acceleration}
\makecell[r]{Maximum acceleration} & \[a_{max}\] & \[m/s^2\] & - & Maximum acceleration of a unit. \\
\hline
\label{programmed_acceleration}
\makecell[r]{Programmed acceleration} & \[a\] & \[m/s^2\] & - & Acceleration used in service. This acceleration must be the best balance between passenger comfort, energy efficiency, and service optimization. \\
\hline
\label{maximum_deceleration}
\makecell[r]{Maximum deceleration} & \[b_{max}\] & \[m/s^2\] & - & Maximum deceleration of a unit during emergency braking. Use absolute value (positive). \\
\hline
\label{programmed_deceleration}
\makecell[r]{Programmed deceleration} & \[b\] & \[m/s^2\] & - & Deceleration used in service. This deceleration/braking must be the best balance between passenger comfort and service optimization. Use absolute value (positive). \\
\hline
\end{longtable}

\begin{longtable}{%
    R{.31\NetTableWidth}%
    C{.08\NetTableWidth}%
    C{.11\NetTableWidth}%
    C{.2\NetTableWidth}%
    L{.3\NetTableWidth}%
}
\hline
\makecell[r]{Definition} & \makecell[c]{Symbol} & \makecell[c]{Unit} & \makecell[c]{Expression} & \makecell[l]{Description} \\
\hline
\hline
\endhead
\label{floor_area_per_standee}
\makecell[r]{Floor area per standee} & \[e\] & \[m^2\] & - & Floor area allocated for each standing place (footprint). Comfortable, adequate circulation: \(0.25 m^2\); Uncomfortable, difficult circulation: \(0.15 m^2\) \\
\hline
\label{vehicle_boarding_door_channels}
\makecell[r]{Number of boarding door channels per vehicle} & \[{n_{Bch}}_y\] & boarding channels & - & Number of channels allowing one person at a time to board the vehicle. \\
\hline
\label{vehicle_alighting_door_channels}
\makecell[r]{Number of alighting door channels per vehicle} & \[{n_{Ach}}_y\] & alighting channels & - & Number of channels allowing one person at a time toalight from the vehicle. \\
\hline
\label{vehicle_mixed_door_channels}
\makecell[r]{Number of mixed door channels per vehicle} & \[{n_{ABch}}_y\] & mixed channels & - & Number of channels allowing one person at a time to either board or alight from the vehicle. \\
\hline
\label{vehicle_door_channels}
\makecell[r]{Number of door channels per vehicle} & \[{n_{ch}}_y\] & access channels & \[{n_{Ach}}_y + {n_{Bch}}_y + {n_{ABch}}_y\] & Number of access channels allowing boarding, alighting, or both in the vehicle. \\
\hline
\label{unit_boarding_door_channels}
\makecell[r]{Number of boarding door channels per unit} & \[{n_{Bch}}_u\] & boarding channels & - & Number of channels allowing one person at a time to board across all vehicles in the unit. \\
\hline
\label{unit_alighting_door_channels}
\makecell[r]{Number of alighting door channels per unit} & \[{n_{Ach}}_u\] & alighting channels & - & Number of channels allowing one person at a time to alight across all vehicles in the unit. \\
\hline
\label{unit_mixed_door_channels}
\makecell[r]{Number of mixed door channels per unit} & \[{n_{ABch}}_u\] & mixed channels & - & Number of channels allowing one person at a time to either board or alight across all vehicles in the unit. \\
\hline
\label{unit_door_channels}
\makecell[r]{Number of door channels per unit} & \[{n_{ch}}_u\] & access channels & \[{n_{Ach}}_u + {n_{Bch}}_u + {n_{ABch}}_u\] & Number of access channels allowing boarding, alighting, or both across all vehicles in theunit. \\
\hline
\end{longtable}


\pagebreak
\subsection*{Lines and paths}

\begin{longtable}{%
    R{.18\NetTableWidth}%
    C{.08\NetTableWidth}%
    C{.04\NetTableWidth}%
    C{.08\NetTableWidth}%
    L{.56\NetTableWidth}%
}
\hline
\makecell[r]{Definition} & \makecell[c]{Symbol} & \makecell[c]{Unit} & \makecell[c]{Expression} & \makecell[l]{Description} \\
\hline
\hline
\endhead
\label{line}
\makecell[r]{Line} & \[L\] & - & - & Transit line comprising one or more distinct paths but uniquely designated with an identifier (number or letter) and a name (the name may change depending on the direction served). A line can be bidirectional, bidirectional loop, or one-way loop, possibly with shortened paths (short line) or extended paths for certain periods of the day or to serve specific clientele. \\
\hline
\label{path}
\makecell[r]{Path} & \[p\] & - & - & Unique path in a determined direction on a line. A path is defined for each distinct stop/node sequence served by the line. A path includes a set of stop-sequences/node-sequences and a geographical route on the network (road, rail, or other network). \\
\hline
\label{path_outbound}
\makecell[r]{Outbound path} & \[p_{out}\] or \[{p^{\prime}}\] & - & - & Path operated in the first direction (outbound) for a bidirectional line or the only direction for a one-way loop line. The choice of outbound direction is arbitrary. Usually, the outbound path is the direction taken during the first departure of the first run of the day on the line. \\
\hline
\label{path_inbound}
\makecell[r]{Inbound path} & \[p_{in}\] or \[{p^{\prime\prime}}\] & - & - & Path operated in the opposite direction (inbound) for a bidirectional line or the second specified direction on the line when the return path is not the exact reverse of the outbound path. \\
\hline
\label{trip}
\makecell[r]{Trip} & \[o\] & - & - & Movement of a service unit on a line path according to a schedule. \\
\hline
\label{schedule}
\makecell[r]{Schedule} & \[O\] & - & - & A set of trips operated for the same service (calendar) for a given line. \\
\hline
\label{run}
\makecell[r]{Run} & \[r\] & - & - & A set of trips served by the same unit or driver. A run can serve multiple paths and multiple lines (interlining). Usually, a run begins and ends at a garage/depot. \\
\hline
\label{block}
\makecell[r]{Block} & \[r_b\] & - & - & A set of consecutive trips served by the same unit within which passengers can remain in the unit for an instant and/or guaranteed transfer. \\
\hline
\label{spatial_directness}
\makecell[r]{Spatial directness} & \[\delta\] & - & \[\frac{d_e}{d_o}\] & Ratio representing the directness of a path compared to the direct as-the-crow-flies path (Euclidean distance). Inverse of tortuosity. \\
\hline
\label{spatial_tortuosity}
\makecell[r]{Spatial tortuosity} & \[\tau\] & - & \[\frac{d_o}{d_e}\] & Ratio representing the tortuosity of a path compared to the direct as-the-crow-flies path (Euclidean distance). Inverse of directness. \\
\hline
\label{spatial_network_directness}
\makecell[r]{Spatial network directness} & \[\delta_n\] & - & \[\frac{d_n}{d_o}\] & Ratio representing the directness of a path compared to the shortest distance path on the network between the two terminals (usually the shortest path by car). Inverse of network tortuosity. \\
\hline
\label{spatial_network_tortuosity}
\makecell[r]{Spatial network tortuosity} & \[\tau_n\] & - & \[\frac{d_o}{d_n}\] & Ratio representing the tortuosity of a path compared to the shortest distance path on the network between the two terminals (usually the shortest path by car). Inverse of network directness. \\
\hline
\label{temporal_directness}
\makecell[r]{Temporal directness} & \[\delta_t\] & - & \[\frac{t_n}{t_o}\] & Ratio representing the directness of a path compared to the fastest travel time of the path on the network between the two terminals (usually the fastest path by car). Inverse of temporal tortuosity. \\
\hline
\label{temporal_tortuosity}
\makecell[r]{Temporal tortuosity} & \[\tau_t\] & - & \[\frac{t_o}{t_n}\] & Ratio representing the tortuosity of a path compared to the fastest travel time of the path on the network between the two terminals (usually the fastest path by car). Inverse of temporal directness. \\
\hline
\end{longtable}

\begin{longtable}{%
    R{.29\NetTableWidth}%
    C{.08\NetTableWidth}%
    C{.12\NetTableWidth}%
    C{.08\NetTableWidth}%
    L{.43\NetTableWidth}%
}
\hline
\makecell[r]{Definition} & \makecell[c]{Symbol} & \makecell[c]{Unit} & \makecell[c]{Expression} & \makecell[l]{Description} \\
\hline
\hline
\endhead
\label{number_of_paths_on_line}
\makecell[r]{Number of paths on line} & \[n_p\] & paths & - & Total number of paths served on a line. \\
\hline
\label{number_of_segments_on_path}
\makecell[r]{Number of segments on path} & \[n_l\] & segments & - & Total number of segments served by the path. \\
\hline
\label{number_of_segments_on_path_outbound}
\makecell[r]{Number of outbound segments} & \[{n_l}^{\prime}\] & segments & - & Total number of segments served by the path in the outbound direction. \\
\hline
\label{number_of_segments_on_path_inbound}
\makecell[r]{Number of inbound segments} & \[{n_l}^{\prime\prime}\] & segments & - & Total number of segments served by the path in the inbound direction. \\
\hline
\label{number_of_nodes_on_path}
\makecell[r]{Number of stops on path} & \[n_q\] or \[n_s\] & stop nodes or stops & - & Total number of stop nodes served by the path. \\
\hline
\label{number_of_nodes_on_path_outbound}
\makecell[r]{Number of outbound stops} & \[{n_q}^{\prime}\] or \[{n_s}^{\prime}\] & stop nodes or stops & - & Total number of stop nodes served by the outbound path. \\
\hline
\label{number_of_nodes_on_path_inbound}
\makecell[r]{Number of inbound stops} & \[{n_q}^{\prime\prime}\] or \[{n_s}^{\prime\prime}\] & stop nodes or stops & - & Total number of stop nodes served by the inbound path. \\
\hline
\label{stop_nodes_linear_density}
\makecell[r]{Stop nodes linear density} & \[{\rho}_{q_L}\] & \[\frac{\textnormal{stop nodes}}{km}\] & \[\frac{n_q}{d}\] & Linear density of stop nodes on a line or line trunk. \\
\hline
\label{stop_nodes_area_density}
\makecell[r]{Stop nodes area density} & \[{\rho}_{q_A}\] & \[\frac{\textnormal{stop nodes}}{{km}^2}\] & \[\frac{N_q}{\textnormal{A}}\] & Area density of stop nodes in a region or given territory. \\
\hline
\label{number_of_outbound_inbound_trips}
\makecell[r]{Number of outbound-inbound trips} & \[N_o\] or \[k\] & trips & - & Total number of trips operated in both directions for a given period on a bidirectional line or in the single direction for a one-way loop line. Care must be taken with ambiguities when multiple distinct paths are used during a period on the line. \\
\hline
\end{longtable}


\pagebreak
\subsection*{Distances and lengths}

\begin{longtable}{%
    R{.3\NetTableWidth}%
    C{.06\NetTableWidth}%
    C{.04\NetTableWidth}%
    C{.14\NetTableWidth}%
    L{.4\NetTableWidth}%
}
\hline
\makecell[r]{Definition} & \makecell[c]{Symbol} & \makecell[c]{Unit} & \makecell[c]{Expression} & \makecell[l]{Description} \\
\hline
\hline
\endhead
\label{acceleration_distance}
\makecell[r]{Acceleration distance} & \[d_a\] & \[m\] & \[\frac{{v_p}^2}{2 a}\] & Distance traveled during the acceleration phase after leaving the previous stop node. \\
\hline
\label{deceleration_distance}
\makecell[r]{Deceleration distance} & \[d_b\] & \[m\] & \[\frac{{v_p}^2}{2 b}\] & Distance traveled during the deceleration/braking phase before reaching the next stop node. \\
\hline
\label{distance_at_programmed_speed}
\makecell[r]{Distance at programmed speed} & \[d_{v_p}\] & \[m\] & \[d_l - d_a - d_b\] (category A or B only) & Distance between the end of acceleration and the beginning of deceleration between stop nodes. \\
\hline
\label{interstop_distance}
\makecell[r]{Interstop distance (segment length)} & \[d_l\] & \[m\] & - & Total distance traveled between two consecutive stop nodes on the path. \\
\hline
\label{average_interstop_distance}
\makecell[r]{Average path interstop distance} & \[\overline{d_l}\] & \[m\] & \[\frac{\sum_{i=1}^{n_l} {d_l}_i} {n_l}\] & Average of interstop distances on the path. \\
\hline
\label{median_interstop_distance}
\makecell[r]{Median path interstop distance} & \[\widetilde{d_l}\] & \[m\] & - & Median of interstop distances on the path. \\
\hline
\label{minimum_interstop_distance}
\makecell[r]{Minimum path interstop distance} & \[{d_l}_{min}\] & \[m\] & - & Length of the shortest interstop segment on the path. \\
\hline
\label{maximum_interstop_distance}
\makecell[r]{Maximum path interstop distance} & \[{d_l}_{max}\] & \[m\] & - & Length of the longest interstop segment on the path. \\
\hline
\label{average_line_interstop_distance}
\makecell[r]{Average line interstop distance} & \[\overline{{d_l}_L}\] & \[m\] & \[\frac{\sum_{j=1}^{n_p} {(\sum_{i=1}^{{n_l}_j} {d_l}_i})} {\sum_{j=1}^{n_p} {{n_l}_j}}\] & Average distance of interstop segments of a line in both directions. \\
\hline
\label{median_line_interstop_distance}
\makecell[r]{Median line interstop distance} & \[\widetilde{{d_l}_L}\] & \[m\] & - & Median of interstop distances on the line. \\
\hline
\label{minimum_line_interstop_distance}
\makecell[r]{Minimum line interstop distance} & \[{{d_l}_L}_{min}\] & \[m\] & - & Length of the shortest interstop segment on the line. \\
\hline
\label{maximum_line_interstop_distance}
\makecell[r]{Maximum line interstop distance} & \[{{d_l}_L}_{max}\] & \[m\] & - & Length of the longest interstop segment on the line. \\
\hline
\label{path_length}
\makecell[r]{Path length} & \[d_p\] & \[m\] & \[\sum_{i=1}^{n_l} {d_l}_i\] & Total length of the path. \\
\hline
\label{path_outbound_length}
\makecell[r]{Outbound path length} & \[d^{\prime}\] & \[m\] & \[\sum_{i=1}^{{n_l}_{p^{\prime}}} {d_l}_i\] & Total length of the path in the outbound direction. \\
\hline
\label{path_inbound_length}
\makecell[r]{Inbound path length} & \[d^{\prime\prime}\] & \[m\] & \[\sum_{i=1}^{{n_l}_{p^{\prime\prime}}} {d_l}_i\] & Total length of the path in the inbound direction. \\
\hline
\label{line_length}
\makecell[r]{Line length} & \[d_L\] & \[m\] & \[\sum_{i=1}^{{N_l}_L} {d_l}_i\] & Total length of the path in one direction. Valid only for unidirectional, loop, or symmetric bidirectional lines. \\
\hline
\label{network_length}
\makecell[r]{Network length} & \[D_{net}\] & \[km\] & \[\sum_{i=1}^{N_{l_u}} {{d_l}_i} + \sum_{i=1}^{N_{l_{m}}} {{d_l}_i}\] & Total length of all unique line segments (in one direction). Count overlapping segments (on which multiple paths run) only once. \\
\hline
\label{lines_length}
\makecell[r]{Lines lengths} & \[D_{L_{tot}}\] & \[km\] & \[\sum_{i=1}^{N_L} {{d_L}_i}\] & Total length of all lines (in one direction). Add overlapping segments. \\
\hline
\label{vehicle_length}
\makecell[r]{Vehicle length} & \[l_y\] & \[m\] & - & Length of a vehicle. \\
\hline
\label{unit_length}
\makecell[r]{Unit length} & \[l_u\] & \[m\] & \[\sum_{i=1}^{n_y} {l_y}_i\] & Length of a unit. \\
\hline
\end{longtable}

\begin{longtable}{%
    R{.3\NetTableWidth}%
    C{.06\NetTableWidth}%
    C{.04\NetTableWidth}%
    C{.14\NetTableWidth}%
    L{.46\NetTableWidth}%
}
\hline
\makecell[r]{Definition} & \makecell[c]{Symbol} & \makecell[c]{Unit} & \makecell[c]{Expression} & \makecell[l]{Description} \\
\hline
\hline
\endhead
\label{network_direct_distance}
\makecell[r]{Direct network distance} & \[d_n\] & \[m\] & - & Distance on the road, rail, or other network that minimizes the total distance between the two terminals, without having to stop en route (does not account for detours made to reach stops). \\
\hline
\label{euclidean_distance}
\makecell[r]{Euclidean distance} & \[d_e\] & \[m\] & - & As-the-crow-flies distance between the two terminals. \\
\hline
\label{access_distance}
\makecell[r]{Access distance} & \[{d_e}_O\] & \[m\] & - & Distance traveled by the user from the origin to the first boarding stop of their transit trip. \\
\hline
\label{egress_distance}
\makecell[r]{Egress distance} & \[{d_e}_D\] & \[m\] & - & Distance traveled by the user from the last stop of their transit trip to the destination. \\
\hline
\label{access_egress_distance}
\makecell[r]{Access/egress distance} & \[{d_e}_{OD}\] & \[m\] & \[{d_e}_O + {d_e}_D\] & Distance traveled by the user from origin to first stop and from last stop to destination of their transit trip. \\
\hline
\label{transfer_distance}
\makecell[r]{Transfer distance} & \[d_{tr}\] & \[m\] & - & Distance traveled to travel from the alighting stop to the boarding stop during a specific transfer. \\
\hline
\label{total_transfer_distance}
\makecell[r]{Total transfer distance} & \[{d_{tr}}_{tot}\] & \[m\] & \[\sum_{i=1}^{n_{tr}} {d_{tr}}_i\] & Total distance traveled between stops during all transfers made during the trip. Does not include access and egress distances. \\
\hline
\label{total_access_egress_transfer_distance}
\makecell[r]{Total access/egress and transfer distance} & \[d_{e_{tot}}\] & \[m\] & \[{d_e}_{OD} + {d_{tr}}_{tot}\]& Sum of access distances from trip origin to first stop and from last stop to trip destination, plus total transfer distances. This distance represents the total distance traveled by walking, cycling, or driving to access transit service at origin and destination and to transfer between lines. \\
\hline
\label{in_vehicle_distance}
\makecell[r]{In-vehicle distance} & \[d_{veh}\] & \[m\] & - & Distance traveled in vehicle on a line between boarding and alighting on the line. \\
\hline
\label{total_in_vehicle_distance}
\makecell[r]{Total in-vehicle distance} & \[d_{{veh}_{tot}}\] & \[m\] & \[\sum_{i=1}^{n_{tr}+1} d_{{veh}_i}\] & Total distance traveled in vehicle during the complete transit trip, including distances of all line segments used. \\
\hline
\label{total_od_distance}
\makecell[r]{Total OD distance} & \[d_{OD}\] & \[m\] & \[d_{e_{tot}} + d_{{veh}_{tot}}\] & Total distance traveled by a userduring a transit trip, including access and transfers. \\
\hline
\label{total_od_transit_distance}
\makecell[r]{Total OD transit distance} & \[d_{transit}\] or \[d_{TC}\] & \[m\] & \[d_{OD}\] & Total distance calculated for the trip using transit. Allows comparison of competitiveness between modes. \\
\hline
\label{total_od_driving_distance}
\makecell[r]{Total OD driving distance} & \[d_{car}\] or \[d_{driving}\]  & \[m\] & - & Total distance of the fastest path calculated by car. Care must be taken to specify if the optimal path was calculated taking congestion into account. \\
\hline
\label{total_od_cycling_distance}
\makecell[r]{Total OD cycling distance} & \[d_{bicycle}\] or \[d_{cycling}\] & \[m\] & - & Total distance of the fastest path calculated by bicycle. \\
\hline
\label{total_od_walking_distance}
\makecell[r]{Total OD walking distance} & \[d_{foot}\] or \[d_{walking}\] & \[m\] & - & Total distance of the fastest path calculated on foot. \\
\hline
\end{longtable}



\pagebreak
\subsection*{Time and durations}
\begin{longtable}{%
    R{.28\NetTableWidth}%
    C{.07\NetTableWidth}%
    C{.03\NetTableWidth}%
    C{.22\NetTableWidth}%
    L{.40\NetTableWidth}%
}
\hline
\makecell[r]{Definition} & \makecell[c]{Symbol} & \makecell[c]{Unit} & \makecell[c]{Expression} & \makecell[l]{Description} \\
\hline
\hline
\endhead
\label{acceleration_time}
\makecell[r]{Acceleration time} & \[t_a\] & \[s\] & \[\frac{v_p}{a}\] & Time required to accelerate from a complete stop to programmed speed. \\
\hline
\label{deceleration_time}
\makecell[r]{Deceleration time} & \[t_b\] & \[s\] & \[\frac{v_p}{b}\] & Time required to slow down and brake from programmed speed to a complete stop. \\
\hline
\label{travel_time_at_programmed_speed}
\makecell[r]{Travel time at programmed speed} & \[t_{v_p}\] & \[s\] & \[\frac{d_{v_p}}{v_p}\] (category A or B only) & Time between the end of acceleration and the beginning of deceleration of a segment between stops. \\
\hline
\label{segment_interstop_travel_time}
\makecell[r]{Segment/Interstop travel time} & \[t_l\] & \[s\] & \[t_a + t_{v_p} + t_b\] or \[\sqrt[]{\frac{2(a + b){d}l}{a b}}\] if the interstop distance doesn't allow reaching programmed speed & Travel time between two consecutive stops that doesn't include dwell times, but includes acceleration and deceleration times, as well as traffic light or congestion stop and go times. The expressions are only valid for categories A or B (no congestion or traffic lights). \\
\hline
\label{average_interstop_travel_time}
\makecell[r]{Average interstop travel time} & \[\overline{t_l}\] & \[s\] & \[\frac{\sum{i=1}^{n_l} {t_l}i}{n_l}\] & Average of interstop travel times of a path or outbound and inbound paths of a bidirectional line. \\
\hline
\label{median_interstop_travel_time}
\makecell[r]{Median interstop travel time} & \[{\widetilde{t_l}}\] & \[s\] & & Median of interstop travel times of a path or outbound and inbound paths of a bidirectional line. \\
\hline
\label{door_opening_time}
\makecell[r]{Door opening time} & \[t_{do}\] & \[s\] & - & Time required for opening all doors of a unit for passenger boarding and/or alighting. \\
\hline
\label{door_closing_time}
\makecell[r]{Door closing time} & \[t_{dc}\] & \[s\] & - & Time required for closing all doors of a unit after passenger boarding and/or alighting. \\
\hline
\label{dwell_time}
\makecell[r]{Dwell time} & \[t_q\] or \[t_s\] & \[s\] & $\begin{gathered}[t] \max{\Big(\frac{n_B}{n_{{Bch}_u}} \overline{t_B}\ ,\ \frac{n_A}{n{{Ach}_u}} \overline{t_A}\Big)} \ + t_{do} + t_{dc} \end{gathered}$ & Time during which the unit in service on a path is at a complete stop, allowing passenger boarding and alighting at the platform or at the stop sign. \\
\hline
\label{average_dwell_time}
\makecell[r]{Average dwell time} & \[\overline{t_q}\] & \[s\] & \[\frac{\sum_{i=1}^{n_q} {t_q}i}{n_q}\] & Average of dwell times on a path. \\
\hline
\label{median_dwell_time}
\makecell[r]{Median dwell time} & \[\widetilde{t_q}\] & \[s\] & & Median of dwell times on a path. \\
\hline
\label{stop_to_stop_time}
\makecell[r]{Stop-to-stop time} & \[t_{\Delta s}\] & \[s\] & \[t_q + t_l\] & Total time between departure from the previous stop and departure from the next stop. Includes interstop travel time and dwell time. \\
\hline
\label{line_average_dwell_time}
\makecell[r]{Line average dwell time} & \[\overline{t_q}_L\] & \[s\] & \[\frac{\sum_{j=1}^{n_p} {(\sum_{i=1}^{{n_q}_j} {{t_q}_j}_i})} {\sum_{j=1}^{n_p} {n_q}_j}\] & Average of dwell times on a line. Care must be taken with ambiguities when the line has multiple paths, particularly bidirectional asymmetric paths with dissimilar dwell times. \\
\hline
\label{maximum_dwell_time}
\makecell[r]{Maximum dwell time} & \[{t_q}_{max}\] & \[s\] & \[\max_{i=1}^{n_q} {{t_q}_i}\] & Dwell time at the busiest node stop and/or requiring the most boarding and alighting time over the entire path. \\
\hline
\label{line_maximum_dwell_time}
\makecell[r]{Line maximum dwell time} & \[{{t_q}_L}_{max}\] & \[s\] & \[\max_{i=1}^{{n_q}_L} {{t_q}_i}\] & Dwell time at the busiest nodestop and/or requiring the most boarding and alighting time on the line. Care must be taken with ambiguities when the line has multiple paths, particularly bidirectional asymmetric paths with dissimilar dwell times. \\
\hline
\label{reaction_time}
\makecell[r]{Reaction time} & \[t_r\] & \[s\] & - & Reaction time of the driver or autonomous system between the start of the braking/deceleration or acceleration command or request and the beginning of braking/deceleration or acceleration. \\
\hline
\label{operating_time}
\makecell[r]{Operating time} & \[T_o\] & \[s\] & \[\sum_{i=1}^{n_l} {t_l}_i + \sum_{i=1}^{n_q - 1} {t_q}_i = \frac{d_L}{V_o}\]& Total time between departure from the origin terminal and arrival at the destination terminal. Usually, one less dwell time than the number of stops on the path is counted (the time at the first stop is included in the layover). \\
\hline
\label{operating_time_outbound}
\makecell[r]{Outbound operating time} & \[{T_o}^\prime\] & \[s\] & \[\sum_{i=1}^{{n_l}_p} {t_l}_i + \sum_{i=1}^{n_q - 1} {t_q}_i \] where p is the outbound path & Total time between departure from the origin terminal and arrival at the destination terminal of the outbound path. Usually, one less dwell time than the number of stops on the path is counted (the time at the first stop is included in the layover). \\
\hline
\label{operating_time_return}
\makecell[r]{Inbound operating time} & \[{T_o}^{\prime\prime}\] & \[s\] & \[\sum_{i=1}^{{n_l}_p} {t_l}_i + \sum_{i=1}^{n_q - 1} {t_q}_i \] where p is the inbound path & Total time between departure from the origin terminal and arrival at the destination terminal of the inbound path. Usually, one less dwell time than the number of stops on the path is counted (the time at the first stop is included in the layover). \\
\hline
\label{layover_time}
\makecell[r]{Layover time} & \[{t_t}\] & \[s\] & - & Time required to alight passengers at the last stop, to allow the unit to become available for the next path, to make up for accumulated delays, and to board passengers for the next departure in the opposite direction. Usually, layover only includes dwell time at terminal departure, but not at arrival (included in operating time). \\
\hline
\label{minimum_layover_time}
\makecell[r]{Minimum layover time} & \[{t_t}_{min}\] & \[s\] & - & Minimum required layover time, regardless of planned operating time. \\
\hline
\label{path_layover_time}
\makecell[r]{Path layover time} & \[{t_t}_p\] & \[s\] & \[\max \Big( {t_t}_{min}\ ,\  {\gamma_t} {T_o} \Big)\] & Layover time at the specific path's arrival terminal. \\
\hline
\label{outbound_layover_time}
\makecell[r]{Outbound layover time} & \[{t_t}^\prime\] & \[s\] & \[\max \Big( {t_t}_{min}\ ,\  {\gamma_t}^\prime {T_o}^{\prime} \Big)\] & Layover time at the departure terminal (at the end of the outbound path). \\
\hline
\label{inbound_layover_time}
\makecell[r]{Inbound layover time} & \[{t_t}^{\prime\prime}\] & \[s\] & \[\max \Big( {t_t}_{min}\ ,\  {\gamma_t}^{\prime\prime} {T_o}^{\prime\prime} \Big)\] & Layover time at the departure terminal (at the end of the inbound path). \\
\hline
\label{total_layover_time}
\makecell[r]{Total layover time} & \[t_t\] & \[s\] & \[\max \Big( 2{t_t}_{min}\ ,\  {\gamma_t}^\prime {T_o}^\prime + {\gamma_t}^{\prime\prime} {T_o}^{\prime\prime} \Big)\] or \[ \max \Big( 2{t_t}_{min}\ ,\  2 \gamma_t T_o \Big)\] for a symmetric line & Total layover time at all terminals of a bidirectional line or loop line. Care must be taken with ambiguities if the line has multiple distinct paths. Layover time includes passenger boarding and alighting times at terminals. \\
\hline
\label{layover_coefficient}
\makecell[r]{Layover coefficient} & \[\gamma_t\] & - & \[\frac{t_t} {T_o}\] & Percentage of operating time used for layovers. Layover time is added to operating time to obtain cycle time. Used for symmetric lines. \\
\hline
\label{outbound_layover_coefficient}
\makecell[r]{Outbound layover coefficient} & \[{\gamma_t}^\prime\] & - & \[\frac{{t_t}^\prime}{{T_o}^\prime}\] & Percentage of outbound path operating time used for outbound layover. \\
\hline
\label{inbound_layover_coefficient}
\makecell[r]{Inbound layover coefficient} & \[{\gamma_t}^{\prime\prime}\] & - & \[\frac{{t_t}^{\prime\prime}}{{T_o}^{\prime\prime}}\] & Percentage of inbound path operating time used for inbound layover. \\
\hline
\label{total_layover_coefficient}
\makecell[r]{Total layover coefficient} & \[\gamma_t\] & - & \[\frac{{t_t}^{\prime} + {t_t}^{\prime\prime}} {{T_o}^{\prime} + {T_o}^{\prime\prime}}\] & Percentage of total operating time used for layovers. \\
\hline
\label{non_productive_tts_layover_time}
\makecell[r]{Non-productive TTS layover time} & \[{t_t}_{h_p}\] & \[s\] & \[T_c - {T_c}_{min}\] & Time that must be added to the minimum cycle time of a line to achieve a clock-face schedule.\\
\hline
\label{total_non_productive_ttslayover_time}
\makecell[r]{Total non-productive TTS layover time} & \[{t_t}_{{h_p}_{total}}\] & \[s\] & \[\sum_{i=1}^{n_L} {t_t}_{{h_p}_i} {{n_u}_i}\] where \(n_L\) is the number of lines affected by the clock-face scheduling & Total time that must be added to the minimum cycle times of the lines to achieve a clock-face schedule.\\
\hline
\label{half_cycle_time}
\makecell[r]{Half-cycle time} & \[{T_c}_p\] & \[s\] & \[{T_o} + {t_t} \] & Total time between departure from the departure terminal and the end of layover at the arrival terminal. \\
\hline
\label{outbound_half_cycle_time}
\makecell[r]{Outbound half-cycle time} & \[{{T_c}_p}^{\prime}\] & \[s\] & \[{T_o}^{\prime} + {t_t}^{\prime} \] & Total time between departure from the departure terminal and the end of layover at the arrival terminal of the outbound path. \\
\hline
\label{inbound_half_cycle_time}
\makecell[r]{Inbound half-cycle time} & \[{{T_c}_p}^{\prime\prime}\] & \[s\] & \[{T_o}^{\prime\prime} + {t_t}^{\prime\prime} \] & Total time between departure from the departure terminal and the end of layover at the arrival terminal of the inbound path. \\
\hline
\label{cycle_time}
\makecell[r]{Cycle time} & \[T_c\] & \[s\] & \[{T_o}^\prime + {t_t}^{\prime} +  {T_o}^{\prime\prime} + {t_t}^{\prime\prime}\] & The cycle time (or unit availability time) includes operating time in both directions and layover time at all terminals of outbound and inbound paths or at the unique terminal in the case of a loop line. This time represents the total time the unit needs to be ready for the next trip. \\
\hline
\end{longtable}


\begin{longtable}{%
    R{.3\NetTableWidth}%
    C{.08\NetTableWidth}%
    C{.04\NetTableWidth}%
    C{.14\NetTableWidth}%
    L{.44\NetTableWidth}%
}
\hline
\makecell[r]{Definition} & \makecell[c]{Symbol} & \makecell[c]{Unit} & \makecell[c]{Expression} & \makecell[l]{Description} \\
\hline
\hline
\endhead
\label{direct_network_travel_time}
\makecell[r]{Direct network travel time} & \[t_n\] & \[s\] & - & Travel time calculated on the fastest road, rail, or other network between the two terminals, without necessarily passing through path stops. \\
\hline
\label{deadhead_time}
\makecell[r]{Deadhead time} & \[t_d\] & \[s\] & - & Sum of travel times between depot and first terminal at start of run, between last terminal and depot at end of run, travel times required for repositioning from terminals, and interlining travel times if the run serves multiple lines. \\
\hline
\label{platform_time}
\makecell[r]{Platform time} & \[T_p\] & \[s\] & \[k T_c + t_d\] where \[k\] is the number of round trips during the run & Total time during which a unit is in operation on a complete run, in productive service and deadheading. Care must be taken with ambiguities when runs include interlining paths. \\
\hline
\label{access_egress_time}
\makecell[r]{Access/egress time} & \[t_{e}\] & \[s\] & - & Travel time by walking, cycling, or driving to reach a stop. \\
\hline
\label{access_time}
\makecell[r]{Access time} & \[{t_e}_O\] & \[s\] & \[v_e {d_e}_O\] & User access time between trip origin and the first stop of their transit trip. A routing engine for the access mode (walking, cycling, driving, or other) can provide a calibrated access time accounting for road, pedestrian or cycling networks and traffic signals. \\
\hline
\label{egress_time}
\makecell[r]{Egress time} & \[{t_e}_D\] & \[s\] & \[v_e {d_e}_D\] & User access time between the last stop of their transit trip and the trip destination. \\
\hline
\label{access_egress_time_od}
\makecell[r]{Access/egress time} & \[{t_e}_{OD}\] & \[s\] & \[{t_e}_O + {t_e}_D\]& Total user access time from trip origin to first stop and from last stop to destination of their transit trip. Does not include total transfer access time. \\
\hline
\label{transfer_time}
\makecell[r]{Transfer time} & \[t_{tr}\] & \[s\] & - & Travel time to move from one stop to another during a transfer. Does not include waiting time during transfer. \\
\hline
\label{total_transfer_time}
\makecell[r]{Total transfer time} & \[{t_{tr}}_{tot}\] & \[s\] & \[\sum_{i=1}^{n_{tr}} d_{tr}\] & Total travel time to move between stops during all transfers made during the transit trip. Does not include access times at origin and destination or waiting time during transfer. \\
\hline
\label{total_access_egress_transfer_time}
\makecell[r]{Total access/egress and transfer time} & \[{t_e}_{tot}\] & \[s\] & \[{t_e}_{OD} + {t_{tr}}_{tot}\] & Sum of user access times from trip origin to first stop and from last stop to trip destination, plus total transfer access time. This time represents the total travel time spent walking, cycling, or driving to access transit service at origin and destination and to travel between transfer stops. \\
\hline
\label{waiting_time}
\makecell[r]{Waiting time} & \[t_w\] & \[s\] & - & Waiting time before boarding at a stop. \\
\hline
\label{origin_waiting_time}
\makecell[r]{Origin waiting time} & \[{t_w}_O\] & \[s\] & - & Waiting time at the first boarding stop of the transit trip. \\
\hline
\label{transfer_waiting_time}
\makecell[r]{Transfer waiting time} & \[{t_w}_{tr}\] & \[s\] & - & Waiting time before boarding at a stop during a transfer. \\
\hline
\label{minimum_waiting_time}
\makecell[r]{Minimum waiting time} & \[{t_w}_{min}\] & \[s\] & - & Minimum safety time (early arrival) before boarding at a stop to account for possible delays of the previous line during a transfer, uncertainties in access times, and the possibility that the unit departs ahead of the scheduled time at the boarding stop. Used for transit path calculations. \\
\hline
\label{total_transfer_waiting_time}
\makecell[r]{Total transfer waiting time} & \[{{t_w}_{tr}}_{tot}\] & \[s\] & \[\sum_{i=1}^{n_{tr}} {{t_w}_i}\] & Total waiting time during transfers. Does not include origin waiting time or transfer access times. \\
\hline
\label{total_waiting_time}
\makecell[r]{Total waiting time} & \[{t_w}_{tot}\] & \[s\] & \[{t_w}_O + {{t_w}_{tr}}_{tot}\] & Total waiting time, including origin waiting time and transfer waiting times. \\
\hline
\label{in_vehicle_time}
\makecell[r]{In-vehicle time} & \[t_{veh}\] & \[s\] & - & Time spent in vehicle on a line between boarding at a stop and alighting at a subsequent stop on the line. \\
\hline
\label{total_in_vehicle_time}
\makecell[r]{Total in-vehicle time} & \[{t_{veh}}_{tot}\] & \[s\] & \[\sum_{i=1}^{n_{tr}+1} t_{{veh}_i}\] & Total time spent in vehicle during the complete transit trip, including travel times of all line segments used and en-route stop times. \\
\hline
\label{total_od_time}
\makecell[r]{Total OD time} & \[T_{OD}\] & \[s\] & \[ {t_e}_{tot} + {t_w}_{tot} + {t_{veh}}_{tot}\] & Total time between origin and destination. \\
\hline
\label{average_single_boarding_time}
\makecell[r]{Average single boarding time} & \[\overline{t_B}\] & \[s\] & - & - \\
\hline
\label{average_single_alighting_time}
\makecell[r]{Average single alighting time} & \[\overline{t_A}\] & \[s\] & - & - \\
\hline
\label{reported_total_time}
\makecell[r]{Reported total time} & \[T_{rep}\] \[{T_{rep}}_L\] & \[s\] & - & Total time reported as worked by employees. Use index L when the coefficient represents the efficiency of a particular line (the line must not share units or workforce with another). \\
\hline
\label{total_paid_time}
\makecell[r]{Total paid time} & \[T_{paid}\] \[{T_{paid}}_L\] & \[s\] & - & Total time paid to employees. Includes reported worked time and absences, delays, leave, etc. Use index L when the coefficient represents the efficiency of a particular line (the line must not share units or workforce with another). \\
\hline
\label{total_service_time}
\makecell[r]{Total service time} & \[T_{serv}\] \[{T_{serv}}_L\] & \[s\] & - & Total time for which service work was performed by employees. This time does not include deadhead times and layover times. Use index L when the coefficient represents the efficiency of a particular line (the line must not share units or workforce with another). \\
\hline
\label{personnel_attendence_coefficient}
\makecell[r]{Personnel attendance coefficient} & \[\eta_a\] \[{\eta_a}_L\] & - & \[\frac{T_{rep}}{T_{paid}}\] \[\frac{{T_{rep}}_L}{{T_{paid}}_L}\] & Used to evaluate the ratio between reported worked hours and paid hours which include absences, delays, leave, etc. Use index L when the coefficient represents the efficiency of a particular line (the line must not share units or workforce with another). \\
\hline
\label{run_cutting_and_schedule_efficiency_coefficient}
\makecell[r]{Run-cutting and schedule efficiency coefficient} & \[\eta_s\] \[{\eta_s}_L\] & - & \[\frac{T_{serv}}{T_{rep}}\] \[\frac{{T_{serv}}_L}{{T_{rep}}_L}\] & Used to evaluate the ratio between hours worked to provide service to users and reported worked hours, which include meetings, breaks, pre-departure checks, deadhead times, terminal times, etc. Use index L when the coefficient represents the efficiency of a particular line (the line must not share units or workforce with another). \\
\hline
\label{terminal_efficiency_coefficient}
\makecell[r]{Terminal efficiency coefficient} & \[\eta_t\] \[{\eta_t}_L\] & - & \[\frac{{T_o}^\prime + {T_o}^{\prime\prime}}{T_c}\] when lines are bidirectional and symmetric & Represents the ratio between total round-trip operating time and cycle time which includes terminal layovers. Use index L when the coefficient represents the efficiency of a particular line (the line must not share units or workforce with another). \\
\hline
\label{global_efficiency_coefficient}
\makecell[r]{Global operating efficiency coefficient} & \[\eta\] \[{\eta}_L\] & - & \[\eta_a \eta_s \eta_t\] \[{\eta_a}_L {\eta_s}_L {\eta_t}_L\] & Represents the global efficiency of a line or network. Use index L when the coefficient represents the efficiency of a particular line (the line must not share units or workforce with another). \\
\hline
\label{total_transit_od_time}
\makecell[r]{Total transit OD travel time} & \[T_{transit}\] or \[T_{TC}\] & \[s\] & \[T_{OD}\] & Time calculated in transit. Allows comparison of competitiveness between modes. \\
\hline
\label{total_driving_od_time}
\makecell[r]{Total driving OD travel time} & \[T_{car}\] or \[T_{driving}\] & \[s\] & - & Travel time of the fastest path calculated by car. Care must be taken to specify if this time and the path used accounts for congestion. \\
\hline
\label{total_bicycle_od_time}
\makecell[r]{Total cycling OD travel time} & \[T_{bicycle}\] or \[T_{cycling}\] & \[s\] & - & Travel time of the fastest path calculated by bicycle. \\
\hline
\label{total_walking_od_time}
\makecell[r]{Total walking OD travel time} & \[T_{foot}\] or \[T_{walking}\] & \[s\] & - & Travel time of the fastest path calculated on foot. \\
\hline
\label{transit_driving_competitivity_coefficient}
\makecell[r]{Transit/driving competitivity coefficient} & \[\mu_{car}\] or \[\mu_{driving}\] & - & \[\frac{T_{transit}}{T_{car}}\] & A value \(> 1\) indicates that transit is slower. \\
\hline
\label{transit_cycling_competitivity_coefficient}
\makecell[r]{Transit/cycling competitivity coefficient} & \[\mu_{bicycle}\] or \[\mu_{cycling}\] & - & \[\frac{T_{transit}}{T_{bicycle}}\] & A value \(> 1\) indicates that transit is slower. \\
\hline
\label{transit_walking_competitivity_coefficient}
\makecell[r]{Transit/walking competitivity coefficient} & \[\mu_{foot}\] or \[\mu_{walking}\] & - & \[\frac{T_{transit}}{T_{foot}}\] & A value \(> 1\) indicates that transit is slower. \\
\hline
\end{longtable}



\pagebreak
\subsection*{Speeds and velocities}

\begin{longtable}{%
    R{.26\NetTableWidth}%
    C{.08\NetTableWidth}%
    C{.04\NetTableWidth}%
    C{.18\NetTableWidth}%
    L{.44\NetTableWidth}%
}
\hline
\makecell[r]{Definition} & \makecell[c]{Symbol} & \makecell[c]{Unit} & \makecell[c]{Expression} & \makecell[l]{Description} \\
\hline
\hline
\endhead
\label{unit_maximum_speed}
\makecell[r]{Unit maximum speed} & \[v_{max}\] & \[{km}/h\] & - & Speed that the unit can reach at maximum power. \\
\hline
\label{design_speed}
\makecell[r]{Design speed} & \[v_d\] & \[{km}/h\] & - & Maximum possible speed on a network section. Takes into account comfort and safety constraints. \\
\hline
\label{legal_speed}
\makecell[r]{Legal speed} & \[v_{reg}\] & \[{km}/h\] & - & Legal speed allowed on the network section. \\
\hline
\label{programmed_speed}
\makecell[r]{Programmed speed} & \[v_p\] & \[{km}/h\] & - & Programmed service speed on a network section. Usually, this speed is determined to optimize, in the best balance, energy efficiency and minimization of interstop travel times. \(v_p \leq v_{reg} \leq v_d\) \\
\hline
\label{operating_speed}
\makecell[r]{Operating speed} & \[V_o\] & \[{km}/h\] & \[\frac{d_p}{T_o}\] & Commercial speed perceived by the user on a path, which includes stop dwell times. \\
\hline
\label{outbound_operating_speed}
\makecell[r]{Outbound operating speed} & \[{V_o}^\prime\] & \[{km}/h\] & \[\frac{d^{\prime}}{{T_o}^{\prime}}\] & Operating speed for the outbound path. \\
\hline
\label{inbound_operating_speed}
\makecell[r]{Inbound operating speed} & \[{V_o}^{\prime\prime}\] & \[{km}/h\] & \[\frac{d^{\prime\prime}}{{T_o}^{\prime\prime}}\] & Operating speed for the inbound path. \\
\hline
\label{line_operating_speed}
\makecell[r]{Line operating speed} & \[{V_o}_L\] & \[{km}/h\] & \[\frac{d^{\prime} + d^{\prime\prime}}{{T_o}^{\prime} + {T_o}^{\prime\prime}}\] & Commercial speed on the line. Care must be taken with ambiguities when the line is not symmetrically bidirectional or loop. \\
\hline
\label{half_cycle_speed}
\makecell[r]{Half-cycle speed} & \[V_{1/2c}\] & \[{km}/h\] & \[\frac{d_p}{T_{1/2c}}\] & Average speed during a half-cycle (path) \\
\hline
\label{outbound_half_cycle_speed}
\makecell[r]{Outbound half-cycle speed} & \[{V_{1/2c}}^{\prime}\] & \[{km}/h\] & \[\frac{d^{\prime}}{{T_{1/2c}}^{\prime}}\] & Average speed during outbound half-cycle time. \\
\hline
\label{inbound_half_cycle_speed}
\makecell[r]{Inbound half-cycle speed} & \[{V_{1/2c}}^{\prime\prime}\] & \[{km}/h\] & \[\frac{d^{\prime\prime}}{{T_{1/2c}}^{\prime\prime}}\] & Average speed during inbound half-cycle time. \\
\hline
\label{cycle_speed}
\makecell[r]{Cycle speed} & \[V_c\] & \[{km}/h\] & \[\frac{d^{\prime} + d^{\prime\prime}}{T_c}\] & Average round-trip speed during cycle time. \\
\hline
\label{platform_speed}
\makecell[r]{Platform speed} & \[V_p\] & \[{km}/h\] & \[\frac{k(d^{\prime} + d^{\prime\prime})}{T_p}\] where \[k\] is the number of round trips during the run & Average speed during the time for which a unit is in operation on a complete run, in productive service and deadheading. Care must be taken with ambiguities when runs include interlining paths. \\
\hline
\label{segment_speed}
\makecell[r]{Segment/interstop speed} & \[v_l\] & \[{km}/h\] & \[\frac{d_l}{t_l}\] & Average speed between departure from the previous stop and arrival at the next stop (segment). Does not include dwell time. \\
\hline
\label{stop_to_stop_speed}
\makecell[r]{Stop-to-stop speed} & \[v_{\Delta s}\] or \[v_{\Delta q}\] & \[{km}/h\] & \[\frac{d_l}{t_l + t_q}\] & Average speed between departure from the previous stop and departure from the next stop. Includes dwell time. \\
\hline
\label{average_segment_speed}
\makecell[r]{Average segment/interstop speed} & \[\overline{v_l}\] & \[{km}/h\] & \[\frac{ \sum_{i=1}^{n_l} {\frac{{d_l}_i}{{t_l}_i}}} {n_l} \] & Average of interstop speeds of segments on a path or line. \\
\hline
\label{average_in_vehicle_speed}
\makecell[r]{Average in-vehicle speed} & \[v_{veh}\] & \[{km}/h\] & - & Average operating speed of line segments or lines used during the trip. \\
\hline
\label{access_egress_speed}
\makecell[r]{Access/egress speed} & \[v_e\] & \[{km}/h\] & - & Walking, cycling, or driving speed to access stops at origin and destination and during transfers. A well-calibrated path calculation provides more realistic speeds, taking into account traffic signals, congestion, and obstacles. \\
\hline
\label{od_speed}
\makecell[r]{OD speed} & \[V_{OD}\] & \[{km}/h\] & \[\frac{d_{OD}}{T_{OD}}\] & Global average speed of a path from origin to destination, taking into account access, transfers, all in-vehicle segments, as well as all stop and waiting times. \\
\hline
\end{longtable}



\pagebreak
\subsection*{Frequencies and periods}

\begin{longtable}{%
    R{.29\NetTableWidth}%
    C{.08\NetTableWidth}%
    C{.05\NetTableWidth}%
    C{.15\NetTableWidth}%
    L{.43\NetTableWidth}%
}
\hline
\makecell[r]{Definition} & \makecell[c]{Symbol} & \makecell[c]{Unit} & \makecell[c]{Expression} & \makecell[l]{Description} \\
\hline
\hline
\endhead
\label{frequency}
\makecell[r]{Frequency} & \[f\] & \[\text{units}/h\] & \[\frac{60}{h}=\frac{n_u}{T_c}\] & Number of passages of units per hour on a route or line (\(T_c\) is in hours). \\
\hline
\label{minimum_frequency}
\makecell[r]{Minimum frequency} & \[f_{min}\] & \[\text{units}/h\] & \[\frac{60}{h_{max}}\] & Minimum service frequency according to national, regional, or local constraints and regulations. This frequency can differ depending on the service period (morning, morning peak, daytime, evening peak, evening, night, weekend, etc.) \\
\hline
\label{minimum_peak_frequency}
\makecell[r]{Minimum peak frequency} & \[{f_{peak}}_{min}\] & \[\text{units}/h\] & \[\frac{60}{{h_{peak}}_{max}}\] & Minimum peak service frequency according to national, regional, or local constraints and regulations. \\
\hline
\label{minimum_off_peak_frequency}
\makecell[r]{Minimum off-peak frequency} & \[{f_{off}}_{min}\] & \[\text{units}/h\] & \[\frac{60}{{h_{off}}_{max}}\] & Minimum off-peak service frequency according to national, regional, or local constraints and regulations. Usually applicable between morning and afternoon peaks and in the evening. \\
\hline
\label{way_constrained_maximum_frequency}
\makecell[r]{Way constrained maximum frequency} & \[{f_l}_{max}\] & \[\text{units}/h\] & \[\frac{60}{{h_l}_{min}}\] & Maximum possible service frequency taking into account the way in which units assigned to the line are travelling. This frequency accounts for maximum permitted speeds, safety criteria, and obstacles on the way (switches, congestion, curve radii, etc.) \\
\hline
\label{dwell_constrained_maximum_frequency}
\makecell[r]{Dwell constrained maximum frequency} & \[{f_s}_{max}\] & \[\text{units}/h\] & \[\frac{60}{{h_s}_{min}}\] & Maximum possible service frequency taking into account the limiting stop node/station on the line. Takes into account available space at stops, number of units that can stop there, planned stop times, and safety criteria. Most often, \({f_s}_{max} \ll {{f_l}_{max}}\) \\
\hline
\label{maximum_frequency}
\makecell[r]{Maximum frequency} & \[f_{max}\] & \[\text{units}/h\] & \[\min \Big({{f_s}_{max}}\ ,\ {{f_l}_{max}}\Big)\] & Maximum possible service frequency taking into account way and stop constraints. \\
\hline
\label{peak_frequency}
\makecell[r]{Peak frequency} & \[f_{peak}\] & \[\text{units}/h\] & \[\frac{60}{h_{peak}}\] & Frequency during the maximum peak period (maximum offered frequency) on a line. \\
\hline
\label{ultra_peak_frequency}
\makecell[r]{Ultra-peak frequency} & \[f_{up}\] & \[\text{units}/h\] & \[\frac{60}{h_{up}}\] & Required frequency during the busiest 15 minutes of the maximum peak period (maximum required frequency) on a line. \\
\hline
\label{off_peak_frequency}
\makecell[r]{Off-peak frequency} & \[f_{off}\] & \[\text{units}/h\] & \[\frac{60}{h_{off}}\] & Frequency during the off-peak period (offered frequency) on a line. \\
\hline
\label{average_frequency}
\makecell[r]{Average frequency} & \[\overline{f}\] & \[\text{units}/h\] & \[\frac{\sum_{i=1}^{n_\text{periods}} {f_i}}{n_\text{periods}}\] & Average frequency of a set of service periods on a line. \\
\hline
\label{stop_weighted_peak_frequency}
\makecell[r]{Stop weighted peak frequency} & \[{f_{peak}}_s\] or \[{f_{peak}}_q\] & \[\text{units}/h\] & \[\frac{\sum_{i=1}^{n_q} {f_i}}{n_q}\] & Specify if the frequency at each stop is the frequency of the line with the best frequency serving the stop or the average frequency of all lines serving the stop. \\
\hline
\label{headway_or_period}
\makecell[r]{Headway or period} & \[h\] & \[min\] & \[\frac{60}{f}=\frac{T_c}{n_u}\] & Time headway between consecutive passages of two units on a line (\(T_c\) is in minutes). \\
\hline
\label{maximum_headway}
\makecell[r]{Maximum headway} & \[h_{max}\] & \[min\] & \[\frac{60}{f_{min}}\] & Maximum service headway according to national, regional, or local constraints and regulations. This headway can differ depending on the service period (morning, morning peak, daytime, evening peak, evening, night, weekend, etc.) \\
\hline
\label{maximum_peak_headway}
\makecell[r]{Maximum peak headway} & \[{h_{peak}}_{max}\] & \[min\] & \[\frac{60}{{f_{peak}}_{min}}\] & Maximum peak headway according to national, regional, or local constraints and regulations. \\
\hline
\label{maximum_off_peak_headway}
\makecell[r]{Maximum off-peak headway} & \[{h_{off}}_{max}\] & \[min\] & \[\frac{60}{{f_{off}}_{min}}\] & Maximum off-peak headway according to national, regional, or local constraints and regulations. Usually applicable between morning and afternoon peaks and in the evening. \\
\hline
\label{way_constrained_minimum_headway}
\makecell[r]{Way constrained minimum headway} & \[{h_l}_{min}\] & \[min\] & \[\frac{60}{{f_l}_{max}}\] & Minimum possible service headway taking into account the way in which units assigned to the line move. This headway accounts for maximum permitted speeds, safety criteria, and obstacles on the way (switches, congestion, curves, etc.) \\
\hline
\label{dwell_constrained_minimum_headway}
\makecell[r]{Dwell constrained minimum headway} & \[{h_s}_{min}\] & \[min\] & \[\frac{60}{{f_s}_{max}}\] & Minimum possible service headway taking into account the limiting stop on the line. Takes into account available space at stops, number of units that can stop there, planned stop times, and safety criteria. Most often, \({h_s}_{min} \gg {{h_l}_{min}}\) \\
\hline
\label{minimum_headway}
\makecell[r]{Minimum headway} & \[h_{min}\] & \[min\] & \[\max \Big({{h_s}_{min}}\ ,\ {{h_l}_{min}}\Big)\] & Maximum possible service frequency taking into account way and stop constraints. \\
\hline
\label{peak_headway}
\makecell[r]{Peak headway} & \[h_{peak}\] & \[min\] & \[\frac{60}{f_{peak}}\] & Interval during the maximum peak period (minimum offered headway) on a line. \\
\hline
\label{ultra_peak_headway}
\makecell[r]{Ultra-peak headway} & \[h_{up}\] & \[min\] & \[\frac{60}{f_{up}}\] & Required headway during the busiest 15 minutes of the maximum peak period (minimum required headway) on a line. \\
\hline
\label{off_peak_headway}
\makecell[r]{Off-peak headway} & \[h_{off}\] & \[min\] & \[\frac{60}{f_{off}}\] & Interval during the off-peak period (offered frequency) on a line. \\
\hline
\label{average_headway}
\makecell[r]{Average headway} & \[\overline{h}\] & \[min\] & \[\frac{\sum_{i=1}^{n_\text{periods}} {h_i}}{n_\text{periods}}\] & Average headway of a set of service periods on a line. \\
\hline
\label{stop_weighted_peak_headway}
\makecell[r]{Stop weighted peak headway} & \[{h_{peak}}_s\] or \[{h_{peak}}_q\] & \[min\] & \[\frac{\sum_{i=1}^{n_q} {h_i}}{n_q}\] & Specify if the headway at each stop is the headway of the line with the smallest headway serving the stop or the average headway of all lines serving the stop. \\
\hline
\label{pulsation_headway_or_tts_headway}
\makecell[r]{Pulsation headway or TTS headway} & \[h_p\] & \[min\] & - & Interval allowing synchronization of multiple lines at focal points (synchronized schedules or Timed-Transfer-System - TTS). Usually: 10, 15, or 20 minutes in urban networks and 30 or 60 minutes in interurban/regional networks. \\
\hline
\end{longtable}



\pagebreak
\subsection*{Volumes and demand}

\begin{longtable}{%
    R{.35\NetTableWidth}%
    C{.08\NetTableWidth}%
    C{.06\NetTableWidth}%
    C{.1\NetTableWidth}%
    L{.41\NetTableWidth}%
}
\hline
\makecell[r]{Definition} & \makecell[c]{Symbol} & \makecell[c]{Unit} & \makecell[c]{Expression} & \makecell[l]{Description} \\
\hline
\hline
\endhead
\label{volume}
\makecell[r]{Volume/Demand} & \[P\] & \[pass/h\] & - & Number of passengers per hour on a path, line, at a stop, stop node, or station. \\
\hline
\label{number_of_boardings_at_stop}
\makecell[r]{Number of boardings at stop/node} & \[n_B\] & \[pass\] & - & - \\
\hline
\label{number_of_alightings_at_stop}
\makecell[r]{Number of alightings at stop/node} & \[n_A\] & \[pass\] & - & - \\
\hline
\label{total_number_of_boardings}
\makecell[r]{Total number of boardings} & \[N_B\] & \[pass\] & \[\sum_{i=1}^{n_q} {{n_B}_i}\] & Total number of boardings on a line trip, in one direction. \\
\hline
\label{total_number_of_alightings}
\makecell[r]{Total number of alightings} & \[N_A\] & \[pass\] & \[\sum_{i=1}^{n_q} {{n_A}_i}\] & Total number of alightings on a line trip, in one direction. \\
\hline
\label{maximum_number_of_boardings_in_channel}
\makecell[r]{Maximum number of boardings in channel} & \[{n_B}_{max}\] & \[pass\] & \[\overline{n_B} \xi_B\] & Total number of boardings at the busiest boarding channel. \\
\hline
\label{maximum_number_of_alightings_in_channel}
\makecell[r]{Maximum number of alightings in channel} & \[{n_A}_{max}\] & \[pass\] & \[\overline{n_A} \xi_A\] & Total number of alightings at the busiest alighting channel. \\
\hline
\label{average_number_of_boardings_per_channel}
\makecell[r]{Average number of boardings per channel} & \[\overline{n_B}\] & \[pass\] & \[\frac{n_B} {{n_{Bch}}_u}\] & Average number of boardings per boarding channel. \\
\hline
\label{average_number_of_alightings_per_channel}
\makecell[r]{Average number of alightings per channel} & \[\overline{n_A}\] & \[pass\] & \[\frac{n_A} {{n_{Ach}}_u}\] & Average number of alightings per alighting channel. \\
\hline
\label{boarding_distribution_coefficient}
\makecell[r]{Boarding distribution coefficient} & \[\xi_B\] & - & \[\frac{{n_B}_{max}} {\overline{n_B}}\] & Factor used to determine the distribution of boardings across all boarding channels. \\
\hline
\label{alighting_distribution_coefficient}
\makecell[r]{Alighting distribution coefficient} & \[\xi_A\] & - & \[\frac{{n_A}_{max}} {\overline{n_A}}\] & Factor used to determine the distribution of alightings across all alighting channels. \\
\hline
\label{maximum_volume}
\makecell[r]{Maximum volume (hourly)} & \[P_{max}\] & \[pass/h\] & - & Total number of passengers who traveled on the busiest segment of a line, in the busiest direction, during the maximum peak hour. \\
\hline
\label{segment_volume}
\makecell[r]{Segment volume} & \[P_l\] & \[pass/h\] & - & Total number of passengers who traveled on the segment, per hour. \\
\hline
\label{average_path_volume}
\makecell[r]{Average path volume} & \[\overline{P_p}\] & \[pass/h\] & \[\frac{\sum_{i=1}^{{n_l}} {P_l}_i}{{n_l}}\] & Average number of passengers per path segment per hour. Specify the period (peak, off-peak, day, etc.) \\
\hline
\label{average_outbound_volume}
\makecell[r]{Average outbound volume} & \[\overline{{P_p}^{\prime}}\] & \[pass/h\] & \[\frac{\sum_{i=1}^{{n_l}^{\prime}} {P_l}_i}{{n_l}^{\prime}}\] & Average number of passengers per outbound path segment per hour. Specify the period (peak, off-peak, day, etc.) \\
\hline
\label{average_inbound_volume}
\makecell[r]{Average inbound volume} & \[\overline{{P_p}^{\prime\prime}}\] & \[pass/h\] & \[\frac{\sum_{i=1}^{{n_l}^{\prime\prime}} {P_l}_i}{{n_l}^{\prime\prime}}\] & Average number of passengers per inbound path segment per hour. Specify the period (peak, off-peak, day, etc.) \\
\hline
\label{average_line_volume}
\makecell[r]{Average line volume} & \[\overline{P_L}\] & \[pass/h\] & \[\frac{\overline{{P_p}^{\prime}} + \overline{{P_p}^{\prime\prime}}}{2}\] & Average number of passengers per line segment per hour. Specify the period (peak, off-peak, day, etc.) \\
\hline
\label{maximum_load_segment}
\makecell[r]{Maximum load segment (MLS)} & \[l_{P_{max}}\] & - & - & Busiest segment of a line, in the busiest direction, during the maximum peak hour. \\
\hline
\label{maximum_volume_stop}
\makecell[r]{Maximum volume stop} & \[s_{P_{max}}\] & - & - & Stop with the highest number of boardings and alightings during the maximum peak hour (single sign or platform). \\
\hline
\label{maximum_volume_stop_node}
\makecell[r]{Maximum volume stop node} & \[q_{P_{max}}\] & - & - & Stop node with the highest number of boardings and alightings during the maximum peak hour. \\
\hline
\label{maximum_volume_station}
\makecell[r]{Maximum volume station} & \[S_{P_{max}}\] & - & - & Station with the highest number of boardings and alightings across all its served stops during the maximum peak hour. \\
\hline
\end{longtable}

\begin{longtable}{%
    R{.25\NetTableWidth}%
    C{.06\NetTableWidth}%
    C{.07\NetTableWidth}%
    C{.21\NetTableWidth}%
    L{.41\NetTableWidth}%
}
\hline
\makecell[r]{Definition} & \makecell[c]{Symbol} & \makecell[c]{Unit} & \makecell[c]{Expression} & \makecell[l]{Description} \\
\hline
\hline
\endhead
\label{ultra_peak_15_minutes_volume}
\makecell[r]{Ultra-peak 15 minutes volume} & \[P_{up15}\] & \[pass/{{15}\ min}\] & - & Total number of passengers who traveled on the busiest segment of a line, in the busiest direction, during the maximum 15-minute peak period. Allows obtaining the most precise peak volume. \\
\hline
\label{ultra_peak_volume}
\makecell[r]{Ultra-peak volume (hourly)} & \[P_{up}\] & \[pass/h\] & \[4 P_{up15}\] & Maximum 15-minute volume extrapolated to an hour. \\
\hline
\label{peak_hour_factor}
\makecell[r]{Peak-hour factor (PHF)} & \[\gamma_{PHF}\] & - & \[\frac{P_{max}}{P_{up}} = \frac{1}{\gamma_{PHC}}\] & Factor used to account for the maximum 15-minute volume. \(0.25 \leq \gamma_{PHF} \leq 1\). \\
\hline
\label{peak_hour_coefficient}
\makecell[r]{Peak-hour coefficient (PHC)} & \[\gamma_{PHC}\] & - & \[\frac{P_{up}}{P_{max}} = \frac{1}{\gamma_{PHF}}\] & Coefficient used to account for the maximum 15-minute volume. \(1 \leq \gamma_{PHC} \leq 4\). \\
\hline
\label{design_volume}
\makecell[r]{Design volume} & \[P_d\] & \[pass/h\] & \[\gamma_{PHC} P_{max} = P_{up}\] & Volume used to determine required line capacities or the capacity utilization coefficient. \\
\hline
\label{daily_path_volume}
\makecell[r]{Daily path volume} & \[{{P_p}_{day}}\] & \[pass/day\] & \[\sum_{i=0}^{23} {P_i}\] where \(P_i\) represents passenger volume between hour \(i\) and hour \(i+1\) on the path & Number of passengers per day transported on the path, in one direction. \\
\hline
\label{daily_line_volume}
\makecell[r]{Daily line volume} & \[{{P_L}_{day}}\] & \[pass/day\] & \[\sum_{i=0}^{23} {P_i}\] where \(P_i\) represents passenger volume between hour \(i\) and hour \(i+1\) & Number of passengers per day transported on all trips of the line. \\
\hline
\label{potential_demand}
\makecell[r]{Potential demand} & \[P_{pot}\] & \[pass/h\] & \[P_{lat} + P_{obs}\] & Passenger volume that would exist if service and pricing were optimal. \\
\hline
\label{latent_demand}
\makecell[r]{Latent demand} & \[P_{lat}\] & \[pass/h\] & \[P_{pot} - P_{obs}\] & Unserved or unsatisfied demand. \\
\hline
\label{observed_volume}
\makecell[r]{Observed volume} & \[P_{obs}\] & \[pass/h\] & \[P_{pot} - P_{lat}\] & Passenger volume observed in the network or line. \\
\hline
\end{longtable}




\pagebreak
\subsection*{Capacities, work and productivity}

\begin{longtable}{%
    R{.25\NetTableWidth}%
    C{.08\NetTableWidth}%
    C{.12\NetTableWidth}%
    C{.15\NetTableWidth}%
    L{.40\NetTableWidth}%
}
\hline
\makecell[r]{Definition} & \makecell[c]{Symbol} & \makecell[c]{Unit} & \makecell[c]{Expression} & \makecell[l]{Description} \\
\hline
\hline
\endhead
\label{capacity}
\makecell[r]{Capacity} & \[C\] & \[places/h\] & - & Line capacity, in one direction, per hour. \\
\hline
\label{unit_capacity}
\makecell[r]{Unit capacity} & \[c_u\] & \[places\] & \[C_{{u}_{se}} + C_{{u}_{st}}\] & Total number of seated and standing places in the unit. \\
\hline
\label{unit_seated_capacity}
\makecell[r]{Unit seated capacity} & \[c_{u_{se}}\] & \[places\] & - & Number of seats in a unit. \\
\hline
\label{unit_standees_capacity}
\makecell[r]{Unit standees capacity} & \[c_{u_{st}}\] & \[places\] & - & Standing capacity of a unit. \\
\hline
\label{vehicle_capacity}
\makecell[r]{Vehicle capacity} & \[c_y\] & \[places\] & \[c_{y_{se}} + c_{y_{st}}\] & Number of seated and standing places in the vehicle. \\
\hline
\label{vehicle_seated_capacity}
\makecell[r]{Vehicle seated capacity} & \[c_{y_{seat}}\] & \[places\] & - & Number of seats in a vehicle. \\
\hline
\label{vehicle_standees_capacity}
\makecell[r]{Vehicle standees capacity} & \[c_{y_{st}}\] & \[places\] & - & Be sure to specify the available space per standing place \(e\) between \(0.15m^2\) (uncomfortable) and \(0.25m^2\) (comfortable, enough space for strollers and some sport equipment). \\
\hline
\label{line_capacity}
\makecell[r]{Line capacity} & \[C_L\] & \[places/h\] & \[c_u f_{up} = \frac{60 c_u}{h_{up}} = \frac{P_d}{\alpha_c}\] & Maximum capacity offered on a line during peak hours, in one direction, taking into account the comfort coefficient. This capacity is \(\leq\) to the maximum theoretical capacity which does not account for desired comfort level. \\
\hline
\label{comfort_coefficient}
\makecell[r]{Comfort coefficient/Load factor} & \[\alpha_c\] & - & - & Coefficient used to determine the required comfort level during design and capacity selection. If standing passengers are to be avoided: \(\alpha_c = c_{{u}_{seat}} / c_u\). Relatively comfortable: \(\alpha_c = 0.7\) \\
\hline
\label{network_maximum_line_capacity}
\makecell[r]{Network maximum line capacity} & \[{C_L}_{max}\] & \[places/h\] & - & Capacity of the line offering the highest capacity in the network, during peak hours, in one direction. \\
\hline
\label{programmed_capacity}
\makecell[r]{Programmed capacity} & \[C_p\] & \[places/h\] & - & Line capacity for a given period (not necessarily peak), in one direction. \\
\hline
\label{line_capacity_coefficient}
\makecell[r]{Line capacity coefficient} & \[\delta_p\] & - & \[\frac{C_p}{C_L}\] & Ratio between programmed capacity for the given period and peak line capacity. \\
\hline
\label{used_capacity_coefficient}
\makecell[r]{Used capacity coefficient} & \[\alpha_u\] & - & \[\frac{\sum_{i=1}^{n_l} {P_{l_i}}}{{C_L}{n_l}}\] & Coefficient used to evaluate the percentage of capacity utilization offered on a line, per hour. \\
\hline
\label{station_capacity}
\makecell[r]{Station capacity} & \[C_S\] & \[places/h\] & - & Number of places that can stop at a station per hour (all lines and directions passing through the station). \\
\hline
\label{node_capacity}
\makecell[r]{Stop node capacity} & \[C_q\] & \[places/h\] & - & Number of places that can stop at a stop node per hour (all lines and directions passing through the node). \\
\hline
\label{stop_capacity}
\makecell[r]{Stop capacity} & \[C_s\] & \[places/h\] & - & Number of places that can stop at a stop (sign or platform) per hour (all lines and directions passing through the stop). \\
\hline
\label{offered_work}
\makecell[r]{Offered work} & \[w_o\] & \[places-km/h\] & \[C_L d_L\] & Work offered on the line in one direction, per hour. Be careful with ambiguities if the line is not symmetrically bidirectional. \\
\hline
\label{outbound_offered_work}
\makecell[r]{Outbound offered work} & \[{w_o}^{\prime}\] & \[places-km/h\] & \[C_L d^{\prime}\] & Be careful with ambiguities if the line is not symmetrically bidirectional. \\
\hline
\label{inbound_offered_work}
\makecell[r]{Inbound offered work} & \[{w_o}^{\prime\prime}\] & \[places-km/h\] & \[C_L d^{\prime\prime}\] & Be careful with ambiguities if the line is not symmetrically bidirectional. \\
\hline
\label{cycle_offered_work}
\makecell[r]{Cycle offered work} & \[{w_{o_c}}\] & \[places-km/h\] & \[{w_o}^{\prime} + {w_o}^{\prime\prime}\] & Total work offered round-trip (complete cycle), per hour. Be careful with ambiguities if the line is not symmetrically bidirectional. \\
\hline
\label{used_work}
\makecell[r]{Used work} & \[w_u\] & \[pass-km/h\] & \[\sum_{i=1}^{n_l} {{P_l}_i {d_l}_i}\] & Work including the volume of passengers who used different segments of the line, in one direction. Be careful with ambiguities if the line is not symmetrically bidirectional. \\
\hline
\label{outbound_used_work}
\makecell[r]{Outbound used work} & \[{w_u}^{\prime}\] & \[pass-km/h\] & \[\sum_{i=1}^{n_l} {{P_l}_i {d_l}_i}\] where \(l_i\) are outbound path segments & Work including the volume of passengers who used different segments of the line, in outbound direction, per hour. Be careful with ambiguities if the line is not symmetrically bidirectional. \\
\hline
\label{inbound_used_work}
\makecell[r]{Inbound used work} & \[{w_u}^{\prime\prime}\] & \[pass-km/h\] & \[\sum_{i=1}^{n_l} {{P_l}_i {d_l}_i}\] where \(l_i\) are inbound path segments & Work including the volume of passengers who used different segments of the line, in inbound direction, per hour. Be careful with ambiguities if the line is not symmetrically bidirectional. \\
\hline
\label{cycle_used_work}
\makecell[r]{Cycle used work} & \[{w_{u_c}}\] & \[pass-km/h\] & \[{w_o}^{\prime} + {w_o}^{\prime\prime}\] & Total work used round-trip (complete cycle), per hour. Be careful with ambiguities if the line is not symmetrically bidirectional. \\
\hline
\label{line_daily_offered_work}
\makecell[r]{Line daily offered work} & \[W_o\] & \[places-km/day\] & \[\sum_{i=1}^{N_o} {w_{o_i}}\] where \(w_{o_i}\) is the work offered per round trip & Total work offered on a line during a complete day of service. Be sure to specify which service type (weekday, weekend, etc.). \\
\hline
\label{line_daily_used_work}
\makecell[r]{Line daily used work} & \[W_u\] & \[pass-km/day\] & \[\sum_{i=0}^{23} {w_{u_i}}\] where \(w_{u_i}\) is the work used per round trip & Total work used by passengers on a line during a complete day of service. Be sure to specify which service type (weekday, weekend, etc.). \\
\hline
\label{network_daily_offered_work}
\makecell[r]{Daily network offered work} & \[W_{o_{net}}\] & \[places-km/day\] & \[\sum_{i=1}^{N_L} {W_{o_i}}\] & Work offered per day across all network lines. \\
\hline
\label{network_daily_used_work}
\makecell[r]{Daily network used work} & \[W_{u_{net}}\] & \[pass-km/day\] & \[\sum_{i=1}^{N_L} {W_{u_i}}\] & Work used per day across all network lines. \\
\hline
\label{productive_capacity}
\makecell[r]{Productive capacity} & \[Q\] & \[places-km/h^2\] & \[C_L {V_o}_L\] & Most useful performance measure, as it accounts for both line capacity and operating speed offered. Productive capacity is per hour, per direction. Be sure to mention the period (peak, off-peak, day, etc.) \\
\hline
\label{unit_daily_productivity}
\makecell[r]{Unit daily productivity} & \[w_u\] & \[unit-km\] & \[N_o (d^{\prime} + d^{\prime\prime})\] where \(N_o\) is the number of round trips per day & Number of km offered by a unit per day. Be sure to specify which service type (weekday, weekend, etc.). \\
\hline
\label{total_unit_daily_productivity}
\makecell[r]{Total unit daily productivity} & \[W_u\] & \[units-km\] & \[\sum_{i=1}^{N_u} {w_{{u}_i}}\] & Total number of unit-km offered per day. Be sure to specify which service type (weekday, weekend, etc.). \\
\hline
\label{productive_volume}
\makecell[r]{Productive volume} & \[Q_u\] & \[pass-km/h^2\] & \[\overline{P_L} {V_o}_L\] & Takes into account average passenger volume on the line. Used productive capacity is per hour, per direction. Be sure to mention the period (peak, off-peak, day, etc.) \\
\hline
\end{longtable}

\begin{longtable}{%
    R{.35\NetTableWidth}%
    C{.08\NetTableWidth}%
    C{.05\NetTableWidth}%
    C{.08\NetTableWidth}%
    L{.44\NetTableWidth}%
}
\hline
\makecell[r]{Definition} & \makecell[c]{Symbol} & \makecell[c]{Unit} & \makecell[c]{Expression} & \makecell[l]{Description} \\
\hline
\hline
\endhead
\label{used_work_coefficient}
\makecell[r]{Used work coefficient} & \[\alpha_w\] & - & \[\frac{w_{u_c}}{w_{o_c}}\] & Used to evaluate the percentage of offered work per hour that was used by passengers. \\
\hline
\label{daily_line_used_work_coefficient}
\makecell[r]{Daily line used work coefficient} & \[\alpha_W\] & - & \[\frac{W_u}{W_o}\] & Used to evaluate the percentage of daily offered work that was used by passengers. \\
\hline
\label{daily_network_used_work_coefficient}
\makecell[r]{Daily network used work coefficient} & \[\alpha_{W_{net}}\] & - & \[\frac{W_{u_{net}}}{W_{o_{net}}}\] & Used to evaluate the percentage of daily offered work that was used by passengers across the entire network. \\
\hline
\end{longtable}



\pagebreak
\subsection*{Costs}

\begin{longtable}{%
    R{.23\NetTableWidth}%
    C{.08\NetTableWidth}%
    C{.04\NetTableWidth}%
    C{.2\NetTableWidth}%
    L{.45\NetTableWidth}%
}
\hline
\makecell[r]{Definition} & \makecell[c]{Symbol} & \makecell[c]{Unit} & \makecell[c]{Expression} & \makecell[l]{Description} \\
\hline
\hline
\endhead
\label{operating_cost_per_unit_hour}
\makecell[r]{Operating cost per unit-hour} & \[c_{t_o}\] & \[\$/h\] & - & Operating cost of a unit per hour of operation. Be sure to include deadhead times and layover times. \\
\hline
\label{operating_cost_per_unit_km}
\makecell[r]{Operating cost per unit-km} & \[c_{d_o}\] & \[\$/km\] & & Operating cost of a unit per km. Be sure to include deadhead distances. \\
\hline
\label{total_operating_cost}
\makecell[r]{Total operating cost} & \[C_o\] & \[\$\] & \[\sum_{i=1}^{N_u} {c_{t_o} t_i}\] or \[\sum_{i=1}^{N_u} {c_{d_o} d_i}\] where \(N_u\) is the number of units in service during the period. & Total operating cost. Be sure to specify the service period for calculating \(t\) and \(d\) (hour, day, week, year, etc.). \\
\hline
\label{total_hourly_operating_cost}
\makecell[r]{Total hourly operating cost} & \[C_{t_o}\] & \[\$\] & \[\sum_{i=1}^{N_u} {c_{t_o}}\] where \(N_u\) is the number of units in service during the period. & Average operating cost per hour. Usually calculated for peak, off-peak or 24h periods. \\
\hline
\label{user_hourly_cost}
\makecell[r]{User hourly cost} & \[c_{t_u}\] & \[\$/h\] & - & Hourly cost associated with user trips (value of time). Can vary based on socio-demographics and user characteristics. \\
\hline
\label{total_user_cost}
\makecell[r]{Total user cost} & \[C_u\] & \[\$\] & \[\sum_{i=1}^{N_{OD}} {c_{t_u} T_{OD}}\] & Total user cost representing the sum of costs for each completed trip. Be sure to specify the service period for calculating \(t\) and \({\Delta s}\) (hour, day, week, year, etc.). \\
\hline
\label{total_hourly_user_cost}
\makecell[r]{Total hourly user cost} & \[C_{t_u}\] & \[\$/h\] & \[\frac{\sum_{i=1}^{N_{OD}} {c_{t_u} T_{OD}}}{\Delta t}\] where \(\Delta t\) is the duration in hours of the period. & Total hourly user cost representing the sum of costs for each completed trip, averaged per hour. Usually calculated for peak, off-peak or 24h periods. \\
\hline
\label{total_cost}
\makecell[r]{Total cost} & \[C\] & \[\$\] & \[C_o + C_u\] & Total cost including operating costs and user time value. \\
\hline
\label{total_hourly_cost}
\makecell[r]{Total hourly cost} & \[C_h\] & \[\$/h\] & \[C_{t_o} + C_{t_u}\] & Total hourly cost including operating costs and user costs. \\
\hline
\end{longtable}
  
\begin{longtable}{%
    R{.3\NetTableWidth}%
    C{.08\NetTableWidth}%
    C{.04\NetTableWidth}%
    C{.08\NetTableWidth}%
    L{.5\NetTableWidth}%
}
\hline
\makecell[r]{Definition} & \makecell[c]{Symbol} & \makecell[c]{Unit} & \makecell[c]{Expression} & \makecell[l]{Description} \\
\hline
\hline
\endhead
\label{minimum_transfer_penalty}
\makecell[r]{Minimum transfer penalty} & \[c_t\] & \[\$\] & - & Penalty (increased time value) for each transfer made. Represents the transfer disutility. \\
\hline
\label{transfer_access_time_penalty_factor}
\makecell[r]{Transfer access time penalty factor} & \[\mu_{tr}\] & - & - & Factor by which transfer access time is multiplied. Varies based on the perceived disutility associated with access times between transfer stops. \\
\hline
\label{waiting_time_penalty_factor}
\makecell[r]{Waiting time penalty factor} & \[\mu_w\] & - & - & Factor by which waiting time is multiplied. Varies based on the perceived disutility associated with waiting times. \\
\hline
\label{access_egress_time_penalty_factor}
\makecell[r]{Access/egress time penalty factor} & \[\mu_{e_{OD}}\] & - & - & Factor by which origin and destination access time is multiplied. Varies based on the perceived disutility associated with origin and destination access times. \\
\hline
\label{total_access_time_penalty_factor}
\makecell[r]{Total access time penalty factor} & \[\mu_e\] & - & - & Factor by which total access time is multiplied. Varies based on the perceived disutility associated with total access time (at origin, destination and during transfers). \\
\hline
\label{in_vehicle_time_penalty_factor}
\makecell[r]{In-vehicle time penalty factor} & \[\mu_{veh}\] & - & - & Factor by which in-vehicle time is multiplied. Varies based on the perceived disutility associated with in-vehicle times. \\
\hline
\end{longtable}



\pagebreak
\subsection*{Agencies, networks, services, schedules, user trips and global indicators}

\begin{longtable}{%
    R{.28\NetTableWidth}%
    C{.05\NetTableWidth}%
    C{.15\NetTableWidth}%
    C{.16\NetTableWidth}%
    L{.37\NetTableWidth}%
}
\hline
\makecell[r]{Definition} & \makecell[c]{Symbol} & \makecell[c]{Unit} & \makecell[c]{Expression} & \makecell[l]{Description} \\
\hline
\hline
\endhead
\label{agency}
\makecell[r]{Agency} & \[A^G\] & - & - & Agency owning a public transit network \\
\hline
\label{network}
\makecell[r]{Network} & \[N^T\] & - & - & Public transit network including lines and services/schedules for one or more agencies \\
\hline
\label{service}
\makecell[r]{Service} & \[H\] &  &  & Set of associated schedules for a given period or service type (weekday service, weekend service, night service, special service, etc.) \\
\hline
\label{scenario}
\makecell[r]{Scenario} & \[H_S\] &  &  & Set of services used for design, comparison and simulation purposes \\
\hline
\label{user_trip}
\makecell[r]{User trip} & \[OD\] & - & - & Transit trip made by a user \\
\hline
\label{total_number_of_user_trips}
\makecell[r]{Total number of user trips} & \[N_{OD}\] & trips & - & Total number of transit trips made by all users. Specify if it's for the entire network, a line, or a route, and if the period is per hour, day, week, year, etc. \\
\hline
\label{direct_user_trips}
\makecell[r]{Total number of direct user trips} & \[N_{{OD}_d}\] & trips & - & Total number of transit trips made by all users without transfers. Specify if it's for the entire network, a line, or a route, and if the period is per hour, day, week, year, etc. \\
\hline
\label{direct_user_trips_coefficient}
\makecell[r]{Direct user trips coefficient} & \[\delta_{{OD}_d}\] & - & \[\frac{N_{{OD}_d}}{N_{OD}}\] & Percentage of trips made directly, without transfers. \\
\hline
\label{urban_population}
\makecell[r]{Urban population} & \[P_{urb}\] & pers. & - & Population of the urban region where the network provides service. \\
\hline
\label{number_of_transfers}
\makecell[r]{Number of transfers} & \[n_{tr}\] & transfers & - & Number of transfers made by the user during a transit trip. \\
\hline
\label{total_number_of_transfers}
\makecell[r]{Total number of transfers} & \[N_{tr}\] & transfers & \[\sum_{i=1}^{} {n_{tr}}_i\] & Total number of transfers made by all users. Specify if it's for the entire network, a line, or a route, and if the period is per hour, day, week, year, etc. \\
\hline
\label{urban_area}
\makecell[r]{Urban area} & \[A_{urb}\] & \[km^2\] & - & Area of the urban region where the network provides service. \\
\hline
\label{transit_usage_habit}
\makecell[r]{Transit usage habit} & \[\overline{n_{OD}}\] & trips & - & Average number of transit trips per person per year, including non-transit users. This indicator is very often used in comparisons between cities but doesn't account for local characteristics, average total trips per person (all modes), and trip chains. \\
\hline
\label{transit_modal_share}
\makecell[r]{Transit modal share} & \[m_{transit}\] & - & - & Percentage of trips in the urban region made by transit (usually for the maximum peak period or an average weekday). \\
\hline
\label{number_of_unique_stop_nodes}
\makecell[r]{Number of unique stop nodes} & \[N_{qu}\] & nodes & - & Total number of stop nodes served by a single line. \\
\hline
\label{number_of_multiple_stop_nodes}
\makecell[r]{Number of multiple stop nodes} & \[N_{qm}\] & nodes & - & Total number of stop nodes served by more than one line. \\
\hline
\label{total_number_of_nodes}
\makecell[r]{Total number of nodes} & \[N_q\] & nodes & \[N_{qu} + N_{qm}\] & Total number of stop nodes in the network. Count stop nodes served by multiple lines (multiple stop nodes) only once. \\
\hline
\label{number_of_lines}
\makecell[r]{Number of lines} & \[N_L\] & lines & - & - \\
\hline
\label{number_of_paths}
\makecell[r]{Number of paths} & \[N_p\] & paths & \[\sum_{i=1}^{N_L} {n_{p_i}}\] & Total number of paths across all network lines. \\
\hline
\label{number_of_unique_segments}
\makecell[r]{Number of unique segments} & \[N_{lu}\] & segments & - & Total number of segments served by a single line. \\
\hline
\label{number_of_multiple_segments}
\makecell[r]{Number of multiple segments} & \[N_{lm}\] & segments & - & Total number of segments served by more than one line. \\
\hline
\label{total_number_of_segments}
\makecell[r]{Total number of segments} & \[N_l\] & segments & \[N_{lu} + N_{lm}\] & - \\
\hline
\label{network_linear_density}
\makecell[r]{Network linear density} & \[\rho_L\] & \[{km}\ \text{of line}/{km}^2\] & \[\frac{d_{net}}{A_{urb}}\] & Network length relative to urban area. \\
\hline
\label{network_linear_coverage}
\makecell[r]{Network linear coverage} & \[\mu_L\] & \[km/\text{pers.}\] & \[\frac{{\Delta s}_{net}}{P_u}\] & Number of kilometers of network per person in the urban area. \\
\hline
\label{line_overlap_coefficient}
\makecell[r]{Line overlap coefficient} & \[\lambda\] & - & \[\frac{N_{lm}}{N_l}\] & Percentage of line segments that serve more than one line. \\
\hline
\label{number_of_depots}
\makecell[r]{Number of depots} & \[N_G\] & depots & - & Number of garages and depots in the network. Be sure to specify the type of depot and the types of units (modes) stored there. \\
\hline
\label{network_average_inter_stop_distance}
\makecell[r]{Network average inter-stop distance} & \[\overline{d_{l_{net}}}\] & \[m\] & \[\frac{d_{net}}{N_l}\] & - \\
\hline
\label{number_of_possible_user_paths}
\makecell[r]{Number of possible user paths} & \[N_{\Delta q}\] & paths & \[\frac{1}{2} N_q (N_q-1)\] & Total number of unique stop node pairs in the network. \\
\hline
\label{number_of_possible_direct_user_paths}
\makecell[r]{Number of possible direct user paths} & \[N_{{\Delta q}_d}\] & paths & \[\frac{1}{2} \Bigg( \sum_{i=1}^{N_L} {n_{q_i}(n_{q_i} - 1)}\] \[-\sum_{j=1}^{N_{Lm}-1} {n_{q_{m_j}}(n_{q_{m_j}} - 1)} \Bigg)\] & Total number of unique stop node pairs in the network that can be served without transfers (direct). \(N_{L_{m}}\) is the number of lines that have at least one common segment with another line and \(n_{q_{m_j}}\) is the number of stations on each segment that each line \(j\) shares with an already counted line. \\
\hline
\label{number_of_possible_user_paths_with_transfers}
\makecell[r]{Number of possible user paths with transfers} & \[N_{{\Delta q}_{tr}}\] & paths & \[N_{\Delta q} - N_{{\Delta q}_d}\] & Total number of unique stop node pairs in the network that can only be served with at least one transfer. \\
\hline
\label{direct_user_paths_coefficient}
\makecell[r]{Direct user paths coefficient} & \[\delta_{{\Delta q}_d}\] & - & \[\frac{N_{{\Delta q}_d}}{N_{\Delta q}}\] & Percentage of possible paths that can be made directly, without transfers. \\
\hline
\label{network_complexity_coefficient}
\makecell[r]{Network complexity coefficient} & \[\beta\] & segments/stop & \[\frac{N_l}{N_q}\] & Ratio of number of segments to number of stop nodes. Don't count bidirectional segments twice. Takes into account both the number of stops per line and the number of multiple segments. \\
\hline
\label{accessible_area}
\makecell[r]{Accessible area} & \[A_a\] & \[km^2\] & - & Area served by the transit network. Obtained using accessibility maps that use a transit calculator. Be sure to specify maximum access distances or times, minimum waiting times before boarding, and the studied period (departure or arrival time). Can be calculated around a departure or arrival point with a departure or arrival time, or globally if no time limit is provided. \\
\hline
\label{network_accessibility_coefficient}
\makecell[r]{Network accessibility coefficient} & \[\mu_a\] & - & \[\frac{A_a}{A_{urb}}\] & Percentage of urban area accessible by transit. \\
\hline
\label{stop_nodes_accessibility_radius}
\makecell[r]{Stop nodes accessibility radius} & \[r_q\] & \[m\] & - & This value is a network analysis parameter. \(r_x\) is usually between 300 and 500m for a bus stop and between 500 and 1000m for a metro station, LRT, BRT, regional train, or other high-reliability and high-frequency mode. For more precision, an access time interval in minutes is preferred but requires a path calculator to obtain access times to the stop. \\
\hline
\label{stop_nodes_accessible_area}
\makecell[r]{Stop nodes accessible area} & \[A_q\] & \[m^2/station\] & \[\pi r_q^2\] & Area served by the transit network. Obtained using accessibility maps that use a transit path calculator. Be sure to specify maximum access distances or times, minimum waiting times before boarding, and the studied period (departure or arrival time). Can be calculated around a departure or arrival point with a departure or arrival time, or globally if no time limit is provided. \\
\hline
\label{network_coverage_coefficient}
\makecell[r]{Network coverage coefficient} & \[\mu_c\] & - & \[\frac{N_q A_q}{A_u}\] if single access radius & Percentage of urban area that includes a transit stop node within the chosen access radius. Clearly indicate the chosen radius, which can vary depending on the type of stop and the level of service offered at each stop. Pay attention to unit conversions (\(km^2 \leftrightarrow m^2\)). \\
\hline
\label{number_of_employees}
\makecell[r]{Number of employees} & \[N_e\] & employees & - & Total number of employees in the network. \\
\hline
\label{labor_productivity}
\makecell[r]{Labor productivity} & \[\eta_{We}\] & \[\text{units}-km/\text{employee}\] & \[\frac{W_u}{N_e}\] & Number of service unit-km offered per employee per day. \\
\hline
\label{labor_efficiency}
\makecell[r]{Labor efficiency} & \[\eta_{Ce}\] & \[\text{places}-km/\text{employee}\] & \[\frac{W_{o_{net}}} {N_e}\] & Work offered per employee per day. \\
\hline
\label{passenger_volume_efficiency}
\makecell[r]{Passenger volume efficiency} & \[\eta_P\] & \[pass/\text{units}-km\] & \[\frac{\sum_{i=1}^{N_L} { \overline{P_{L_{{tot}_i}}}}} {W_u}\] & Volume of passengers transported across all lines per day per unit-km. \\
\hline
\label{energy_consumption}
\makecell[r]{Energy consumption} & \[E\] & \[kWh\] & - & Total energy consumption. Specify the period (per day, per week, per year, etc.) \\
\hline
\label{energy_efficiency}
\makecell[r]{Energy efficiency} & \[\eta_E\] & \[kWh/\text{units}-km\] & \[\frac{E}{W_u}\] & Energy consumption per unit-km. Specify the period (per day, per week, per year). \\
\hline
\label{reliability}
\makecell[r]{Reliability} & \[R\] & \[\%\] & - & Percentage of on-time stop arrivals compared to planned schedules. Ideally: between 0 and 3 minutes late. In practice usually: between 1 minute early and 5 minutes late. \\
\hline
\label{trips_completion_rate}
\makecell[r]{Trips completion rate} & \[R_o\] & \[\%\] & - & Percentage of completed trips compared to planned trips. \\
\hline
\end{longtable}

\end{document}
